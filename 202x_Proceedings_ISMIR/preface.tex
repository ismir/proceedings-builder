\section*{Preface}

Welcome to ISMIR 2023, the 24th International Society for Music Information Retrieval Conference. ISMIR is the world’s leading research forum on processing, searching, organizing, and accessing music-related data. Our community reflects a diversity of scientific disciplines, seniority levels, professional affiliations, and cultural backgrounds. We aim to foster and stimulate this diversity, leading to better science and better music services. The organizing team, who came together from all over the world to ensure the success of this event, welcomes you to ISMIR 2023. 

\subsection*{Scientific Program}

The ISMIR 2023 scientific program comprised three keynote talks, six tutorials and 103 papers. A total of 272 abstracts were registered on the submission system, of which 229 were submitted as complete papers eligible for review. In keeping with the practices of the previous years, a two-tier double-blind review process was conducted involving a total of 211 reviewers and 63 meta-reviewers. Each paper was assigned to a single meta-reviewer and three reviewers, and replacement reviewers were found when the originally assigned reviewer was unable to complete their review. Meta-reviewers were also instructed to complete a full review of each of their assigned papers, in addition to the final meta-review summarizing the individual reviews. Each meta-reviewer and reviewer was responsible for no more than 4 papers, in order that the reviewing load would be manageable, thus promoting careful and thorough reviews. The initial reviewing phase was followed by a discussion period, in which reviewers and meta-reviewers could discuss and revise their assessments of each paper. Meta-reviewers were then instructed to summarize the discussion and reviews in the final report. The Scientific Program Chairs (SPC) made the final decisions on each paper, based on the recommendations of metareviewers and reviewers. 104 papers were accepted (one of which was later withdrawn by the authors), giving an \textbf{acceptance rate of 45.4\%} (or 38.2\% if incomplete submissions are included).  The SPC would like to express their thanks to the ISMIR community of reviewers and metareviewers for their wholehearted support of this critical aspect of a successful ISMIR technical program.

Table \ref{tab:subjects} summarizes the number of submitted and accepted papers in each subject area (as selected by authors during the submission process) together with the corresponding proportion of papers in the program. Table \ref{tab:stats} summarizes the publication statistics over the 24-year history of the conference.


\begin{table}[ht]
\caption{Papers submitted and accepted by subject area\label{tab:subjects}}
\centering
\begin{tabular}{l c c c}
\toprule
\textbf{Subject Area} & \textbf{Submitted} & \textbf{Accepted} & \textbf{Accepted \%} \\
\midrule
MIR tasks & 77 & 21 & 20.2\% \\
Musical features and properties & 42 & 17 & 16.3\% \\
Knowledge-driven approaches to MIR & 37 & 18 & 17.3\% \\
Applications & 35 & 11 & 10.6\% \\
MIR fundamentals and methodology & 22 & 10 & 9.6\% \\
Evaluation, datasets, reproducibility & 19 & 7 & 6.7\% \\
Human-centered MIR & 19 & 8 & 7.7\% \\
Computational musicology & 12 & 6 & 5.8\% \\
MIR and ML for musical acoustics & 5 & 3 & 2.9\% \\
Philosophical and ethical discussions & 4 & 3 & 2.9\%  \\
\midrule
\textbf{Total} & \textbf{272} & \textbf{104} &  \\
\bottomrule
\end{tabular}
\end{table}

%\begin{figure}
%\caption{Number of papers accepted with at least one contributing author from each region}
% \includegraphics[]{}
%\end{figure}


\begin{table}[ht]
\caption{Summary of publication statistics over the 24-year-history of the ISMIR conference\label{tab:stats}}
\centering
\begin{tabular}{ l l c c c c c c c}
\toprule
\textbf{Year} & \textbf{Location} & \textbf{Oral} & \textbf{Poster} & \textbf{Total} & \textbf{Authors} & \textbf{Unique Authors} & $\frac{\textnormal{\textbf{Authors}}}{\textnormal{\textbf{Paper}}}$ & $\frac{\textnormal{\textbf{Unique Authors}}}{\textnormal{\textbf{Paper}}}$ \\
\midrule
2000 & Plymouth & 19 & 16 & 35 & 68 & 63 & 1.9 & 1.8 \\
2001 & Indiana & 25 & 16 & 41 & 100 & 86 & 2.4 & 2.1 \\
2002 & Paris & 35 & 22 & 57 & 129 & 117 & 2.3 & 2.1 \\
2003 & Baltimore & 26 & 24 & 50 & 132 & 111 & 2.6 & 2.2 \\
2004 & Barcelona & 61 & 44 & 105 & 252 & 214 & 2.4 & 2.0 \\
2005 & London & 57 & 57 & 114 & 316 & 233 & 2.8 & 2.0 \\
2006 & Victoria & 59 & 36 & 95 & 246 & 198 & 2.6 & 2.1 \\
2007 & Vienna & 62 & 65 & 127 & 361 & 267 & 2.8 & 2.1 \\
2008 & Philadelphia & 24 & 105 & 105 & 296 & 253 & 2.8 & 2.4 \\
2009 & Kobe & 38 & 85 & 123 & 375 & 292 & 3.0 & 2.4 \\
2010 & Utrecht & 24 & 86 & 110 & 314 & 263 & 2.0 & 2.4 \\
2011 & Miami & 36 & 97 & 133 & 395 & 322 & 3.0 & 2.4 \\
2012 & Porto & 36 & 65 & 101 & 324 & 264 & 3.2 & 2.6 \\
2013 & Curitiba & 31 & 67 & 98 & 395 & 236 & 3.0 & 2.4 \\
2014 & Taipei & 33 & 73 & 106 & 343 & 271 & 3.2 & 2.6 \\
2015 & Málaga & 24 & 90 & 114 & 370 & 296 & 3.2 & 2.6 \\
2016 & New York & 25 & 88 & 113 & 341 & 270 & 3.0 & 2.4 \\
2017 & Suzhou & 24 & 73 & 97 & 324 & 248 & 3.3 & 2.6 \\
2018 & Paris &   &   & 104 & 337 & 265 & 3.2 & 2.5 \\
2019 & Delft &   &   & 114 & 390 & 315 & 3.4 & 2.8 \\
2020 & Virtual &   &   & 115 & 426 & 343 & 3.7 & 3.0 \\
2021 & Virtual &   &   & 104 & 334 & 269 & 3.2 & 2.6 \\
2022 & Bengaluru & & & 113 & 423 & 355 & 3.8 & 3.0 \\
2023 & Milan & & & 103 & 374 & 311 & 3.6 & 3.0 \\
\bottomrule
\end{tabular}
\end{table}

\newpage
\subsubsection*{Best Paper Awards}

Awards for the best paper and best student paper were given at ISMIR 2023. Best paper candidates were selected from the 104 accepted papers. The SPC selected six candidate papers based on reviewers’ and meta-reviewers’ nominations as well as the paper review scores and comments. The final selections were made by specially appointed judges drawn from experienced MIR researchers who had no conflict of interest with any of the award candidates. In addition, judges from Universal Music Group selected one paper from a longer list of highly ranked papers on the basis of its contribution to responsible MIR research, for a special Responsible Research Award, sponsored by UMG.

The following papers were nominated for consideration for the Best Paper Awards (in order of paper number):

\begin{itemize}

\item\emph{Exploring the Correspondence of Melodic Contour with Gesture in Raga Alap Singing},
Shreyas M Nadkarni, Sujoy Roychowdhury, Preeti Rao and Martin Clayton 

\item\emph{CLaMP: Contrastive Language-Music Pre-training for Cross-Modal Symbolic Music Information Retrieval},
Shangda Wu, Dingyao Yu, Xu Tan and Maosong Sun

\item\emph{BPS-Motif: A Dataset for Repeated Pattern Discovery of Polyphonic Symbolic Music},
Yo-Wei Hsiao, Tzu-Yun Hung, Tsung-Ping Chen and Li Su

\item\emph{PESTO: Pitch Estimation with Self-Supervised Transposition-Equivariant Objective},
Alain Riou, Stefan Lattner, Gaëtan Hadjeres and Geoffroy Peeters

\item\emph{LP-MusicCaps: LLM-Based Pseudo Music Captioning},
Seungheon Doh, Keunwoo Choi, Jongpil Lee and Juhan Nam

\item\emph{Singing Voice Synthesis Using Differentiable LPC and Glottal-Flow-Inspired Wavetables},
Chin-Yun Yu and George Fazekas

\end{itemize}

Each of the Best Paper candidates will be invited to publish an extended version of their paper in the Transactions of the International Society for Music Information Retrieval (TISMIR), the open access journal of the Society. The Society will cover the article processing charges of these publications. The following three awards were given:

\paragraph*{Best Paper Award}

\emph{PESTO: Pitch Estimation with Self-Supervised Transposition-Equivariant Objective}, 
Alain Riou, Stefan Lattner, Gaëtan Hadjeres and Geoffroy Peeters

\paragraph*{Best Student Paper Award}

\emph{CLaMP: Contrastive Language-Music Pre-training for Cross-Modal Symbolic Music Information Retrieval}, 
Shangda Wu, Dingyao Yu, Xu Tan and Maosong Sun

\paragraph*{UMG Award for Responsible Research}

\emph{Data Collection in Music Generation Training Sets: A Critical Analysis},
Fabio Morreale, Megha Sharma and I-Chieh Wei


\subsection*{Diversity \& Inclusion}
The ISMIR 2023 conference took a broad view of Diversity, Equity, and Inclusion (DEI). We considered two ``sides'' of DEI: on the one hand, the diversity and equity in the musical objects we study or create, as well as the musical artists, technological approaches, and content that we work with;
and the diversity, equity, and inclusion of the people doing the research in MIR.
From this standpoint, the DEI Chairs, in consultation with the organizing committee, coordinated a variety of initiatives with the aim of bringing together (and widening) the range of perspectives, traditions, and people across our MIR community.
Notably, thanks to the Board and the generous support of our sponsors, we were able to support an unprecedented level of financial support covering travel, accommodation, registration, and childcare costs.
Waivers and fee reductions for the above-mentioned categories were prioritized for underrepresented individuals including women, ethnic minorities, members of the LGBTQIA community, and attendees from low-income countries. In addition, priority was also given to unaffiliated attendees, ``new-to-ISMIR'' presenters, and students. 
All attendees were eligible to apply for childcare grants.


\paragraph*{Inclusion Panel}

The aim of the DEI panel was to foster discussion, both between panelists, and between the audience and the panelists, relating to not only the diversity of people who work in MIR, but also diversity in topics, approaches, and data. 
Panelists were selected based on their excellent track records of commitment to diversity in MIR. After a brief overview of their research and how it has been influenced by, or relates to, DEI, our panelists contributed to a rich discussion of both the content and the people in MIR, and the ways in which we, as a community, can improve.
Discussion topics included: bias in music access, consumption, and recommender systems; barriers and issues of DEI in MIR scholarship and how to overcome them; the relation of DEI and ``ethical AI''; and how to increase the number of women and minorities in our field.


\textbf{Moderator:} Claire Arthur, Georgia Institute of Technology \\ 
\textbf{Panelists:} Anja Volk, Utrecht University; Jin Ha Lee, University of Washington; Christine Bauer, University of Salzburg; Lorenzo Pocarno, European Commission Joint Research Center (Milan)

\paragraph*{Inclusion Meetup}

A DEI meetup session was planned with the aim of designing a social event that would encourage the intermingling of people from different backgrounds, levels of scholarship, and communities. 
Several interactive activities were organized to encourage movement, mingling, discussion, and entertainment.
According to those who who were in attendance (over 100 people!), the event was a success.

\textbf{Host: Riccardo Giampiccolo} 


\subsubsection*{Women in Music Information Retrieval (WiMIR)}

Women in Music Information Retrieval (WiMIR) is a group of people dedicated to promoting the role of, and increasing opportunities for, women in the MIR field. WiMIR’s initiatives started as informal gatherings around breakfast or lunch during ISMIR conferences (2011–2014), and moved to formal WiMIR events included in the conference program (2015–today) garnering a high turnout of both women and allies. These events provide occasions for people to network and to discuss several important issues ranging from mentorship and conference support, to improving the representation of women and, more broadly, diversity in the community. In 2018, WiMIR started hosting its own workshop as a satellite event, in which attendees of all genders participated. These workshops aim to offer participants an opportunity for networking, put the spotlight on technical work done by women in the field, and foster collaboration between women and allies by proposing group work led by project guides to try to solve small research problems or to undertake new research projects that could lead to longer-term collaborations. In 2023, due to the decline (and rotation) in volunteers, the WiMIR workshop initiative was suspended. However, the aim is to resume these very popular and successful workshops again in 2024. The ISMIR 2023 DEI Chairs gratefully acknowledge the support of this year’s WiMIR sponsors, whose contributions support women in the field as well as the broader DEI efforts of this year’s conference.


\subsubsection*{Newcomer Initiative}

A mentoring program was offered in 2023 for prospective authors who are new to ISMIR. They were given feedback on their ideas and drafts of their ISMIR submissions. We would like to thank the following people who volunteered to be mentors for this initiative:
\begin{multicols}{3}\raggedcolumns
\begin{itemize}\setlength{\parskip}{0pt}
    \item Emmanouil Benetos
    \item Geoffroy Peeters
    \item Brian McFee
    \item Juhan Nam
    \item Chris Donahue
    \item Cheng-i Wang
    \item Cory McKay
    \item Jin Ha Lee
    \item Gus Xia
    \item Mitsunori Ogihara
    \item Andre Holzapfel
\end{itemize}
\end{multicols}

\subsection*{Special Sessions}

The Scientific Program Chairs organized two special sessions.  Brief introductions and session information are provided below: 

\subsubsection*{Panel session: Hybrid deep learning for MIR}
    
In MIR, as in many other domains, there is a significant trend towards purely data-driven approaches aimed at directly solving the machine learning problem at hand, while only crudely considering the nature and structure of the data being processed.  In the music domain, prior knowledge can relate to the production of sound (using an acoustic model), the way music is perceived (based on a perceptual model), or how music is composed (using a musicological model).

These models can usually be encoded with only a few parameters, leading to controllable and interpretable systems that can be exploited in modern neural-based machine learning frameworks, resulting in so-called hybrid deep learning models.

The aim of this panel was to illustrate the concept of hybrid deep learning with some specific examples in MIR, and to discuss its limits, merits and potential for future machine learning based music applications.

\textbf{Moderator:} Gaël Richard, Télécom Paris \\
\textbf{Panelists:} George Fazekas, Queen Mary University of London; Changhong Wang, Télécom Paris, Zhiyao Duan, University of Rochester; Gus Xia, NYU Shanghai/MBZUAI

\subsubsection*{Industry Panel}

The industrial panel aimed to facilitate a high-level discussion, providing conference participants with insights into MIR efforts by ISMIR's sponsoring companies. This initiative aimed to foster collaborations among participants, be they from the industry or academic realm. The primary focus was on delving into the future of multi-modal AI in music research -- a burgeoning paradigm that leverages diverse data types like audio, image, text, and speech to enhance outcomes. Each panelist briefly presented their perspective on the topic, leading to an open discussion. Notably, the discourse also explored the relevance of existing Large Language Models and their impact on the field of music.

\textbf{Moderator:} Xavier Serra, Director of the Music Technology Group of the Universitat Pompeu Fabra.\\
\textbf{Panelists:}
Justin Salamon, Senior Research Scientist at Adobe;
Elio Quinton, VP, Artificial Intelligence at Universal Music Group;
Akira Maezawa, Senior Engineer at Yamaha; 
Fabien Gouyon, Senior Director of Research at Pandora -- SiriusXM; 
Romain Hennequin, Head of Research at Deezer; 
Maria Stella Tavella, Senior AI Engineer and Manager at Musixmatch; 
Filip Korzeniowski, Lead Data Scientist at Moises.AI


\subsection*{Late Breaking/Demo Session}

The Late Breaking/Demo (LBD) Session is where we showcase cool works that are still in the making --- prototypes, early ideas and results that generate excitement in the MIR community. This year we received more submissions than expected. To handle the demand, we split the session into two parts and papers were presented in-person as well as using the virtual platform. Following a light review process by the LBD chairs, we accepted 40 papers for live, in-person presentation, while another 8 papers were accepted for virtual presentation, ensuring broader accessibility to the valuable insights shared within the LBD Session. This decision allowed us to accommodate the diverse preferences and circumstances of our contributors and attendees. Following previous years’ practice, LBD contributions are not part of the official ISMIR proceedings and should be seen as non-refereed works — think of them as fresh works in progress and exciting fun demos that led to an exceptionally lively session contributing to this year’s edition of ISMIR. 


\subsection*{Unconference}

ISMIR 2023 reintroduced the ``Unconference'' session, where participants team up into  small groups to engage in discussions on Music Information Retrieval (MIR) topics of their specific interest.
Two weeks prior to the session, participants were invited to propose their preferred topics.
The session then started with a brief plenary in which session topics of greatest interest were selected. 
The session gathered around 80 participants, including both students and senior researchers from academia and industry.
Four topics, namely "MIR + music education," "Open review for MIR," "Human-centered AI," and "Evaluation of generative AI," were chosen for the initial round of discussions. 
Subsequently, participants were divided into four groups, engaging in impromptu discussions for 30 minutes.
Although it is customary in the Unconference to choose new topics every 30 minutes, the participants in each group were so engrossed in their discussions that they extended the conversation into a second and even a third round, all centered around the same topics.
The session concluded with a plenary session, during which one representative from each group was invited to provide a summary of their discussions for the benefit of the other groups. Following this, participants were encouraged to continue their conversations beyond the session and explore opportunities for potential collaborations on the discussed topics.

\subsection*{Music Session}

The ISMIR 2023 Music Session received nine submissions, six of which were accepted for presentation. Half of the contributions were performed live, whereas the remaining ones were pre-recorded and reproduced during the session. All the pre-recorded contributions included video content. 
Overall, the selected submissions showed a high degree of variety, ranging from AI-assisted performances to improvisations with augmented instruments and combinations with visual arts.

Here is the complete list of music pieces that were presented at ISMIR 2023:


\begin{description}
\item[Conversations with our Digital Selves: the development of an autonomous music improviser] Matthew Yee-King, Mark d'Inverno

\item[``confluyo yo, el ambiente me sigue''] Hugo Flores Garcia

\item[Sliogán: a performance composed for the HITar] Andrea Martelloni, Andrew McPherson, Mathieu Barthet

\item[The Words I Tried to Say] Angela Weihan Ng

\item[Nor Hope] Wenbin Lyu

\item[AI Pianist Performance: Collaboration with Soprano Sumi Jo] Taegyun Kwon, Joonhyung Bae, Jiyun Park, Jaeran Choi, Hyeyoon Cho, Yonghyun Kim, Dasaem Jeong, Juhan Nam
\end{description}


\subsection*{Satellite Events }

In addition to the main conference, four satellite events took place immediately before or after ISMIR, and were attended by many ISMIR delegates:

\begin{itemize}
    \item Sound Demixing Workshop, November 4, 2023
    \item Workshop on Reading Music Systems (WoRMS), November 4, 2023
    \item Workshop on Human-Centric Music Information Research (HCMIR), November, 10, 2023
    \item 10th International Conference on Digital Libraries for Musicology (DLfM), November, 10, 2023
\end{itemize}


\subsection*{Acknowledgements}

We are happy to present to you the proceedings of ISMIR 2023. The conference program was made possible thanks to the hard work of many people, including the ISMIR 2023 conference chairs, ISMIR Board members, volunteers, and the many reviewers and meta-reviewers from the program committee.

We would also like to thank our sponsors, whose contributions made this conference possible:

\textit{Platinum sponsors}
\begin{itemize}\setlength{\parskip}{0pt}
    \item Moises
\end{itemize}

\textit{Gold sponsors}
\begin{multicols}{3}\raggedcolumns
\begin{itemize}\setlength{\parskip}{0pt}
    \item Yamaha
    \item MusixMatch
    \item Algoriddim
    \item Deezer
    \item Adobe
    \item Google Research
    \item Universal Music Group
\end{itemize}
\end{multicols}

\textit{Silver sponsors}
\begin{multicols}{3}\raggedcolumns
\begin{itemize}\setlength{\parskip}{0pt}
    \item ACRCloud
    \item Steinberg
    \item Native Instruments
    \item SiriusXM
\end{itemize}
\end{multicols}

We would like to thank the sponsors that explicitly chose to sponsor WiMIR, its grants, and its initiatives:

\textit{Patron}
\begin{itemize}\setlength{\parskip}{0pt}
    \item Deezer
\end{itemize}

\textit{Contributors}
\begin{multicols}{3}\raggedcolumns
\begin{itemize}\setlength{\parskip}{0pt}
    \item Moises
    \item Google Research
    \item Native Instruments
    \item SiriusXM
\end{itemize}
\end{multicols}

\textit{Supporters}
\begin{itemize}\setlength{\parskip}{0pt}
    \item Steinberg
\end{itemize}

ISMIR 2023 would not have been possible without the exceptional contributions of our community in response to our call for participation. The biggest acknowledgment goes to you, the researchers, presenters and participants. 

Paolo Bestagini\\
Simon Dixon\\
Beici Liang\\
Gaël Richard\\
\textbf{Scientific Program Chairs}

Augusto Sarti\\
Fabio Antonacci\\
Mark Sandler\\
\textbf{General Chairs}
