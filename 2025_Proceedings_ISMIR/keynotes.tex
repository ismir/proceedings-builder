\section*{Keynote Speakers}
\addcontentsline{toc}{section}{Keynote Speakers}

\subsection*{Keynote 1: Culture Technology in the Age of AI}

\textbf{Speaker:} Soo Man LEE\\
\textit{Key Producer \& Visionary Leader at A2O Entertainment}

\textbf{Time:} Monday, September 22, 1pm KST

\textbf{Abstract:}
This keynote addresses the significance and future vision of the convergence of culture and technology in driving the global growth of K-Pop. The speaker, founder of SM Entertainment and A2O Entertainment, pioneered the concept of ``Culture Technology (CT)'', which has served as the foundation for the worldwide spread of Korean popular culture. CT systematized the entire value chain—from artist discovery and training to content production and distribution—establishing a framework for artist and IP creation that enabled the rise of K-Pop, while also integrating emerging technologies into music and content to transform the entertainment industry. Through ``Zalpha Pop,'' the speaker envisions a world where humanity connects beyond borders and languages, and foresees the emergence of a prosumer-driven ecosystem in which fans actively participate as creators rather than remaining passive consumers. In a future where artificial intelligence and other advanced technologies accelerate the transformation of the entertainment landscape, CT will serve as a catalyst to amplify human creativity and as the cornerstone of a culturally enriched future where culture and technology thrive in harmony.

\textbf{Biography:}
Born in 1952, Soo Man LEE made his musical debut as part of the duet April and May (\begin{CJK}{UTF8}{mj}4월과 5월\end{CJK}) in 1971. He began his solo career in 1974 and was honored with the MBC Top 10 Singer Award in 1977. In 1978, he earned his bachelor's degree from Seoul National University. In 1980, he became a prominent media figure, hosting MBC Radio's Starry Night (\begin{CJK}{UTF8}{mj}별이 빛나는 밤에\end{CJK}) and the MBC TV talk show With Soo-man Lee (\begin{CJK}{UTF8}{mj}이수만과 함께\end{CJK}). He later pursued graduate studies in the United States, earning a master's degree in Computer Engineering from California State University, Northridge in 1985. In 1989, he founded SM Planning, which evolved into SM Entertainment Co., Ltd. in 1995. Under his leadership, SM debuted its first idol group, H.O.T., in 1996, followed by the debut of solo artist BoA in 2000. He went on to produce and debut leading K-pop acts such as TVXQ, Girls' Generation, EXO, NCT, and aespa, shaping the foundation of the modern K-pop industry.

\subsection*{Keynote 2: Crying, Language, and Song in Korea: From Folk Songs to Pansori}

\textbf{Speaker:} Hey-Jung Kim\\
\textit{Musicologist \& Professor at Gyeongin National University of Education}

\textbf{Time:} Wednesday, September 24, 1pm KST

\textbf{Abstract:}
Koreans cry in the mode of their local folk music. When they are sad and unable to accept a situation, they cry using only certain notes. As they gradually come to terms with reality and reach acceptance or compromise, additional notes are introduced. These shifts in tone reflect emotional transitions embedded in the act of crying. Among traditional Korean musical genres, pansori is a form in which a solo performer narrates an epic story while expressing a wide range of human emotions through music. In pansori, emotional textures—such as denial, anger, sadness, and acceptance—are musically distinct and vividly realized. Listeners who share the same musical and cultural context often experience deep emotional resonance and immersion. This presentation examines how Korean music conveys emotion through the musical language of crying. By analyzing the musical and linguistic elements associated with different emotional states, we explore how sound operates as a shared emotional language in Korean tradition.

\textbf{Biography:}
Dr. Hey-Jung Kim (Professor, Master of Music, Ph.D. in Literature) is a musicologist and the current president of the Pansori Society. She previously served as president of the Society for Korean Folk Songs. In her role as professor at Gyeongin National University of Education, she is dedicated to training future elementary music educators while simultaneously working to preserve and revitalize traditional Korean culture. Since the 1990s, Dr. Kim has documented and analyzed orally transmitted musical traditions such as Korean folk songs (minyo), nongak (farmer's music), and shamanic rituals.

\section*{Special Sessions}
\addcontentsline{toc}{section}{Special Sessions}

\subsection*{WIMIR Session}
Widening Inclusion in MIR (WIMIR) session reporting ongoing efforts to broaden diversity and inclusiveness in the MIR community.

\textbf{WIMIR 2025 Organizers:}\\
Zafar Rafii, Francesca Ronchini, Yun-Ning (Amy) Hung

\subsection*{Special Session on Asian Traditional Music}
Introducing research on Asian traditional music from musicological and anthropological perspectives.

\textbf{Speakers:}

\textit{Sounding Taiwanese through Gramophone Recordings, 1895–1945}\\
Ying-fen Wang, Distinguished Professor at the Graduate Institute of Musicology, National Taiwan University

\textit{Exploring the regularities and evolutionary foundation of music through songs and speech across cultures}\\
Yuto Ozaki, Senior Researcher of Keio Research Institute at Shonan Fujisawa Campus, Keio University

\textit{Applying MIR to Chinese Guqin Tablature Research: Prospects, Challenges, and Opportunities}\\
YU, Hui, Changjiang Distinguished Professor of the Chinese Ministry of Education at Nanjing Normal University

\subsection*{MIREX 2025}

After a three-year hiatus, MIREX (Music Information Retrieval Evaluation eXchange) returned in 2025 with renewed focus on both traditional and modern MIR tasks. This year's competition included 12 evaluation tasks ranging from classic challenges like Audio Chord Estimation and Cover Song Identification to emerging areas like Music Reasoning QA and Song Deepfake Detection.

MIREX 2025 also introduced a new "Call for Challenges" initiative, inviting the community to propose novel research challenges and volunteer as task captains. The competition culminated with poster presentations at the ISMIR Late-Breaking/Demo session, where top-performing teams showcased their systems.

\textbf{MIREX 2025 Organizers:}\\
Junyan Jiang (New York University), Gus Xia (MBZUAI), Akira Maezawa (Yamaha), Ziyu Wang (New York University), Yixiao Zhang (ByteDance Inc.), Ruibin Yuan (Hong Kong University of Science and Technology), J. Stephen Downie (University of Illinois)

\subsection*{RenCon 2025: Expressive Performance Rendering Competition}

RenCon (Performance Rendering Contest) is an international challenge for researchers and developers to submit systems capable of rendering expressive musical performances from symbolic scores. With a rich history dating back to 2002, RenCon has been a platform for showcasing automated music systems at conferences like SMC, NIME, ICMPC, and IJCAI.

This year's competition focused on solo piano performances and was structured in two phases:

\textbf{Phase 1 – Preliminary Round (Online):} Participants submitted performances of assigned pieces (chosen from works by Handel, Beethoven, Rachmaninoff, and Amy Beach) and one free-choice piece, along with a technical report. The submission period ran from May 30 to August 25, 2025.

\textbf{Phase 2 – Live Contest at ISMIR:} Top systems from the preliminary round were invited to render a surprise piece live at ISMIR 2025 in real time. The live contest was open to all ISMIR attendees and the general public, with the audience voting for their favorite system. The winner was announced at the end of the conference.

\textbf{RenCon Organizers:}\\
Huan Zhang (Task Captain, Queen Mary University of London), Taegyun Kwon (Venue Coordinator, KAIST), Junyan Jiang (New York University), Simon Dixon (Queen Mary University of London), Gus Xia (MBZUAI), Akira Maezawa (Yamaha)

\subsection*{Industry Sessions}

\textbf{Industry Meetup}\\
\textit{Monday, September 22, 17:30–19:00, Auditorium Lawn}\\
Moderator: Jaehun Kim

The Industry Meetup connected industry participants with the broader ISMIR community. The event featured "Industry Lightning Talks," where up to 25 industry participants gave one-minute introductions to their organizations, products, or research, fostering networking and collaboration opportunities.

\textbf{Industry Session}\\
\textit{Tuesday, September 23, 13:00–14:15, Auditorium}\\
Moderator: Akira Maezawa

The Industry Session showcased ISMIR 2025 sponsors and provided insights into cutting-edge industrial Music Information Retrieval research and products. Presenting sponsors included: Suno, Yamaha, Universal Music Group, Steinberg, Neutune, Algoriddim, Spotify, Udio, Google DeepMind, Adobe, and MusicAI.
