\section*{Tutorials}
\addcontentsline{toc}{section}{Tutorials}

\textit{Morning Session (9:00--12:30 KST)}

\subsection*{T1: Differentiable Physical Modeling Sound Synthesis}

\textbf{Presenters:} Jin Woo Lee, Stefan Bilbao, and Rodrigo Diaz

\textbf{Abstract:} This tutorial highlights how differentiable physical modeling opens new avenues for musical sound synthesis by combining the interpretability and realism of physical simulation with the learning capacity of modern neural networks. The session covers digital synthesis history, finite difference time domain (FDTD) methods, neural architectures, and differentiable modeling for parameter estimation. Attendees will engage with theoretical material, practical demonstrations, and programming exercises.

\subsection*{T2: Self-supervised Learning for Music}

\textbf{Presenters:} Julien Guinot, Alain Riou, Yuexuan Kong, Marco Pasini, Gabriel Meseguer Brocal, Stefan Lattner

\textbf{Abstract:} This tutorial offers a comprehensive overview of self-supervised learning (SSL) methods in MIR, covering contrastive learning, masked modeling, clustering, and teacher-student approaches. The second half explores generative representation learning and equivariance, illustrating how transformation-aware models can enhance various music processing tasks.

\subsection*{T3: PsyNet: Online Research Platform for Music Studies}

\textbf{Presenters:} Peter Harrison, Harin Lee, Manuel Anglada-Tort, Pol van Rijn, Nori Jacoby

\textbf{Abstract:} This tutorial introduces PsyNet, an online research platform for collecting large, high-quality datasets of human responses to music. The platform supports evaluation tasks, production tasks, and human-in-the-loop procedures, with seamless integration with cloud services and crowdsourcing platforms.

\textit{Afternoon Session (14:00--17:30 KST)}

\subsection*{T4: Differentiable Alignment Techniques for Music Processing}

\textbf{Presenters:} Meinard Müller \& Johannes Zeitler

\textbf{Abstract:} This tutorial introduces differentiable alignment techniques that enable models to learn from weakly aligned data. Beginning with Dynamic Time Warping (DTW), it covers Soft-DTW and Connectionist Temporal Classification (CTC) loss, with applications in multi-pitch estimation, transcription, score-audio alignment, and cross-version retrieval.

\subsection*{T5: Explainable AI for Music Information Retrieval}

\textbf{Presenters:} Valerie Krug, Maral Ebrahimzadeh, Tia Bolle, Jan-Ole Perschewski, Sebastian Stober

\textbf{Abstract:} This tutorial addresses the need to understand AI decision-making in music. As deep learning models achieve state-of-the-art performance, explainable AI (XAI) techniques have emerged as crucial tools for interpreting model behavior. The tutorial provides theoretical foundations and practical exercises for applying XAI techniques to music.

\subsection*{T6: MIR for Health, Medicine, and Well-being}

\textbf{Presenters:} Anja Volk, Elaine Chew, Michael A. Casey

\textbf{Abstract:} This tutorial explores opportunities to employ MIR methods for music, health, medicine, and well-being. Topics include MIR for music therapy, music heart theranostics, and neurology and music information in epilepsy research, connecting MIR with interdisciplinary collaborations in healthcare.

\section*{Satellite Events}
\addcontentsline{toc}{section}{Satellite Events}

\subsection*{Saturday, September 20, 2025}

\textbf{HCMIR25: 3rd Workshop on Human-Centric Music Information Research}

\textit{Where:} Room \#3229, Paik Nam June Hall N25 Building, KAIST\\
\textit{When:} 14:00 -- 18:00

Music and technology have long intertwined, transforming how we create, share, and experience music. The 3rd edition of the HCMIR workshop explores MIR's ethical, societal, and human-centred dimensions. How can we ensure MIR systems are inclusive, ethical, and aligned with human values? This workshop invites researchers, practitioners, and artists to engage in a multidisciplinary discussion.

\textbf{Keynote:} Understanding the Human Experience of Music to Shape Future Technologies of MIR\\
Prof. Kyung Myun Lee, KAIST

\textit{For further details:} \url{https://sites.google.com/view/hcmir25/home}

\subsection*{Friday, September 26, 2025}

\textbf{DLfM 2025: 12th International Conference on Digital Libraries for Musicology}

\textit{Location:} Gabriel Hall, Sogang University, Seoul, South Korea\\
\textit{Satellite event of:} ISMIR 2025

The International Conference on Digital Libraries for Musicology (DLfM) presents a venue for those working on, and with, digital library systems and content in the domain of music and musicology. DLfM welcomes contributions related to any aspect of digital libraries and musicology, including musical archiving and retrieval, cataloguing, musical databases, music encodings, computational musicology, or the application of MIR to musicology.

\textbf{Programme Chair:} Elsa De Luca, CESEM, Universidade Nova de Lisboa\\
\textbf{General Chair:} David M. Weigl, mdw – University of Music and Performing Arts Vienna\\
\textbf{Local Chair:} Dasaem Jeong, Sogang University

\textit{For further details:} \url{https://dlfm.web.ox.ac.uk/12th-international-conference-digital-libraries-musicology}

\textit{For further details:} \url{https://dlfm.web.ox.ac.uk/}

\subsection*{Friday, September 26, 2025}

\textbf{LLM4MA: Large Language Models for Music \& Audio}

\textit{Location:} Jung Geun Mo Conference Hall (5F), KAIST, Daejeon, Korea\\
\textit{Time:} 8:30 am -- 5:00 pm\\
\textit{Online:} \url{https://zoom.us/j/99541677917}

LLM4MA explores the rapidly evolving intersection of large language models (LLMs) and music/audio understanding and generation. The workshop provides a forum for discussing advances in tokenization, long-context modeling, multimodal alignment, and controllability in music applications. It fosters early-stage research and community exchange on emerging methods, challenges, and ethical considerations in AI-driven music creation.

\textbf{Keynote:} Science of AI and AI for Science\\
Prof. Noah A. Smith, University of Washington \& Allen Institute for AI

\textbf{Organizing Committee:} Chenghua Lin, SeungHeon Doh, Liumeng Xue, Ilaria Manco, Gus Xia, and others
