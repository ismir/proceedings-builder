\section*{Preface}

\subsection*{Conference Theme: Harmony of Tradition and Modernity}

ISMIR 2025 embraced the theme "Harmony of Tradition and Modernity," encouraging diverse perspectives on how MIR can bridge past and present. We welcomed research exploring the multifaceted intersections of tradition and innovation—from the preservation and analysis of traditional music forms to the study of contemporary music trends powered by computational methods and data. The conference advanced the understanding of music as a dynamic and evolving cultural force, engaging with both its rich historical roots and its ever-expanding horizons.

\textbf{Conference Logo:} The ISMIR 2025 logo draws inspiration from \textit{Ilwol-obongdo}, the royal folding screen that traditionally stood behind the Korean throne, depicting five peaks with the sun and moon symbolizing cosmic balance. The design integrates modern architectural elements from Daejeon—the Hanbit Tower, Expo Bridge, and KAIST's signature blue color—representing technological advancement. Musical innovation is embodied by the pedal-up symbol, a musical notation marking clear transition points. The color duality of blue (representing academic rigor) and red (representing creative energy) reflects the conference's theme of harmonizing tradition with modernity. The logo was designed by Joonhyung Bae, KAIST.

\subsection*{Messages from the General Chairs}

It is our great pleasure to welcome you to the 26th Conference of the International Society for Music Information Retrieval (ISMIR 2025). The ISMIR conference is the world's leading forum for research on processing, searching, organizing, and accessing music-related data. This year's edition takes place in Daejeon, Korea, from September 21 to 25, 2025, and is jointly organized by the Korea Advanced Institute of Science and Technology (KAIST), Sogang University, and the Korean Society for Music Informatics.

We are delighted to present the ISMIR 2025 program. This year, we received 324 abstracts, from which 278 papers were reviewed. Of these, 99 papers were accepted with an acceptance rate of 35.6\% (35.84\% in ISMIR 2024). As in previous years, the review process was conducted under a double-blind, two-tier model, involving 269 reviewers and 74 meta-reviewers (up from 256 reviewers and 70 meta-reviewers in 2024). Each paper received at least three reviews, including one from a meta-reviewer. We are deeply grateful to all reviewers and meta-reviewers for their time, expertise, and dedication. The accepted papers, authored by 413 authors (328 unique authors), were presented in both oral and poster sessions, with a mean of 4.17 authors per paper (median: 4; max: 10).

Guided by this year's special theme, \textit{Harmony of Tradition and Modernity}, the program features two keynote talks: a legendary K-pop producer and a Korean traditional music expert. It also includes a special session introducing research on Asian traditional music from musicological and anthropological perspectives, an industry session showcasing cutting-edge music services from our sponsors, and a WIMIR session reporting ongoing efforts to broaden diversity and inclusiveness in the MIR community. Evening events include a music program that demonstrates creative applications of MIR technologies in musical works, a concert of Korean traditional music, the ever-popular jam session, and the RenCon challenge, which evaluates systems capable of rendering expressive musical performances from symbolic scores. In addition, we prepare K-Culture Night, a social event where participants can experience Korean traditional games, food, costumes, and music.

Continuing ISMIR's well-established traditions, the program also offers a full day of tutorials, a half day of late-breaking/demo and unconference sessions, and three satellite events before and after the main conference: the Workshop on Human-Centric Music Information Research (HCMIR25), the International Conference on Digital Libraries for Musicology (DLfM), and the Workshop on Large Language Models for Music \& Audio (LLM4MA).

We would like to express our sincere gratitude to our 22 ISMIR sponsors and 3 WIMIR sponsors, raising \$110,000 in support. We are particularly delighted to welcome 9 first-time ISMIR sponsors. Our sponsors represent a truly global partnership: 10 from Asia, 7 from America, and 5 from Europe. Specifically, we thank: Adobe, Moises, Udio, Algoriddim, Google DeepMind, Spotify, Yamaha, Neutune, Steinberg, AlphaTheta, Suno, Universal Music Group, Cochl., AudibleMagic, BMAT, Deezer, Gaudio, MIPPIA, Roland, AudAI, Piascore, and Neutone. Their generous support makes ISMIR 2025 possible. We also gratefully acknowledge the local financial support of the Korea Tourism Organization and the Daejeon Tourism Organization. The conference was supported by the Ministry of Education of the Republic of Korea and the National Research Foundation of Korea (NRF-2024S1A5C3A03046168). Finally, we thank the organizing committee for their dedication, professionalism, and tireless work in bringing this conference to life.

We are not only organizers but also people who truly love ISMIR and its community. What we have learned, experienced, and shared with so many colleagues at ISMIR has shaped us into the researchers we are today. It is a joy for us to return the passion, inspiration, and kindness we have received from ISMIR over the years. Every ISMIR has always remained in our hearts as a joyful and inspiring time, and we have prepared this year's event with the hope that it will be remembered in the same way for you. We warmly invite you to enjoy the very first ISMIR in Korea to the fullest!

\subsection*{Conference Statistics}

ISMIR 2025 brought together 596 participants: 489 full on-site attendees, 32 single-day on-site attendees, and 75 online participants. With virtual attendees from 45 countries, supported by dedicated virtual volunteers and chairs, the conference fostered a truly global community. Over 600 members joined our Slack workspace for real-time communications, and numerous viewers followed the proceedings on YouTube and engaged with the conference program through MiniConf (ismir2025program.ismir.net). Interactive tutorial sessions were held via Zoom, demonstrating the conference's far-reaching impact across continents.

\subsection*{Scientific Program}

The scientific program received 324 abstract submissions, from which 278 papers were reviewed. Of these, 99 papers were accepted with an acceptance rate of 35.6\% (35.84\% in ISMIR 2024). The review process was conducted under a double-blind, two-tier model, involving 269 reviewers and 74 meta-reviewers (up from 256 reviewers and 70 meta-reviewers in 2024). Each paper received at least three reviews, including one from a meta-reviewer. The 99 accepted papers were authored by 413 authors (328 unique authors), with a mean of 4.17 authors per paper (median: 4; max: 10). Accepted papers were presented in both oral and poster sessions throughout the conference.

\subsection*{Grants Program}

The conference supported equitable access through a comprehensive grants program, distributing 99 registration waivers (84 on-site, 15 virtual), 44 accommodation grants, and 5 travel grants. Grants were awarded across four categories: Paper Authors, WIMIR (including Accessibility and Childcare), Music Authors, and LBD/Satellite Events, enabling participation from students, underrepresented groups, and researchers from low- or middle-income countries.

\subsection*{Late-Breaking/Demo Session}

The Late-Breaking/Demo (LBD) session provided a platform for showcasing innovative preliminary work in MIR. With a capacity of 75 posters accepted on a first-come-first-served basis, the session offered an accessible entry point for newcomers and early-career researchers to present prototypes, datasets, and initial concepts, fostering community engagement and feedback.

\subsection*{Diversity, Equity, and Inclusion}

ISMIR 2025 prioritized inclusive participation through multiple initiatives. The conference maintained a Code of Conduct with clear reporting channels, provided accessibility accommodations including mobility and sensory support, and implemented a photo consent policy using yellow lanyards to indicate "do not photograph me" preferences. These efforts, coordinated across Registration, Local Organization, and Virtual teams, ensured a safe and welcoming environment for all attendees.

\subsection*{New-to-ISMIR Paper Mentoring Program}

The New-to-ISMIR Paper Mentoring Session was held to enhance accessibility and encourage participation from a broader, more diverse community. Ten mentees, new to ISMIR, received pre-submission guidance on paper writing and research direction from senior researchers who volunteered their time as past Meta-Reviewers.

We sincerely thank the following mentors for their valuable service: Alexander Lerch, Ajay Srinivasamurthy, Brian McFee, Cheng-i Wang, Chris Donahue, Cory McKay, Ethan Manilow, Jordan B. L. Smith, LEVE Florence, Taegyun Kwon (Emergency Mentor).

\subsection*{Newcomer Squad}

The Newcomer Squad, coordinated by Seungheon Doh, matched 52 newcomers with 13 squad leaders, forming sub-clusters based on research interests. An ISMIR onboarding session was held to help newcomers navigate the conference and connect with the community.

We sincerely thank the following squad leaders for their dedication: Ajay Srinivasamurthy, Vincent Lostanlen, Anja Volk, Sergio Oramas, J. Stephen Downie, Patricia Hu, Jin Ha Lee, Stefan Balke, Lele Liu, Bruno Di Giorgi, Jan Hajič jr., Jordan B. L. Smith, Jingwei Zhao, and Gabriel Meseguer Brocal.

\subsection*{Evening Events and Special Programs}

The conference featured several evening events that complemented the scientific program. The Music Program demonstrated creative applications of MIR technologies in musical works. A Korean Traditional Music Concert showcased traditional Korean music heritage. The ever-popular Jam Session brought together musicians from the community for informal performances. K-Culture Night offered participants an immersive experience of Korean traditional games, food, costumes, and music. The Industry Session showcased cutting-edge music services from our sponsors, providing insights into real-world applications of MIR technologies.

\vspace{1cm}

\noindent Juhan Nam, Dasaem Jeong, and Keunwoo Choi\\
General Chairs of ISMIR 2025
